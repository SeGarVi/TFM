%Copyright (C) 2014 Sergio García Villalonga.

El principal objectiu d'aquest estudi es analitzar l'impacte de la presència d'altres usuaris a l'entorn en la precisió d'un sistema de posicionament en interiors suportat en tecnologia WiFi. Per la consecució d'aquesta fita, podem establir els següents objectius complementaris:

\begin{itemize}
    \item Presentar un estudi de les diferents variables que poden afectar negativament a la precisió en la localització en interiors.
    \item Presentar un estudi de les diferents alternatives tecnològiques a l'hora d'implementar un sistema de localització en interiors.
    \item Investigar les possibles eines lliures que puguin donar suport a l'objectiu principal de la investigació.
    \item Implementar un sistema lliure de localització en interiors mitjançant Wi-Fi en un entorn amb diferents nivell de presència humana intentant reutilitzar qualque eina lliure existent.
    \item Dissenyar un pla de proves que permeti comparar la precisió del sistema implementat en diferents contextos.
    \item Presentar un estudi sobre com l’augment de persones presents afecta a la precisió de la localització del sistema desenvolupat.
    \item Proposar una millora del sistema dissenyat, més tolerant a les aglomeracions.
\end{itemize}
