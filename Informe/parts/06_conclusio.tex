% Copyright (C) 2014 Sergio García Villalonga.
% This work is under the CC-BY-SA license.
% The LICENSE file contains a complete description of your rights.

En l'estudi desenvolupat s'ha creat un sistema de localització en interiors aprofitant aplicacions lliures existents. A partir de la creació d'un mapa de ràdio i de dos mapes de proves s'ha estudiat la influència de la presència d'altres usuaris en la precisió del sistema utilitzant quatre algorismes diferents.

El primer a destacar es que les xarxes \textit{WiFi} permeten crear sistemes de localització en interiors d'una manera senzilla i barata. De fet, ha permès crear una en un entorn aliè a la nostra propietat en un entorn no controlat. De totes maneres, malgrat aquest avantatge, el fet de que la tecnologia \textit{WiFi} no estigui dissenyada amb l'objectiu de localitzar en interiors, fa  que efectes com la fluctuació del senyal, la reflexió o altres, afegeixin una gran imprecisió en les estimacions.

En el nostre cas, amb imprecisions al voltant dels 10 metres, el sistema resultant pot no ser suficient depenent de l'aplicació a la qual es vulgui orientar. Per exemple, amb aquesta imprecisió seria gairebé inútil fer-lo servir a una aplicació destinada a oferir publicitat de la tenda davant la qual es troba l'usuari. En canvi, si els comerços es trobassin agrupats per tipus d'activitat comercial (per exemple, si tots els restaurants es situassin en la mateixa zona), una aplicació podria integrar el sistema de localització aquí presentat i oferir promocions diverses.

Depenent del tipus d'algoritme s'observen comportaments diferents en la localització dins botigues i passadissos. En el cas de \textit{KNN} i \textit{WKNN}, la localització és més precisa en passadissos que dins botigues. En el cas de \textit{MAP} i \textit{MMSE} la precisió és major en tendes, encara que les diferències de mesures botiga-passadissos són menys acusades que en el cas de \textit{KNN} i \textit{WKNN}. Malgrat que els valors d'error mitjà en \textit{MAP} i \textit{MMSE} arriben a ser molt similars en tots els casos, l'error mitjà en passadissos plens és especialment dolenta. En general, els millors resultats en quant a la precisió en passadissos s'obtenen amb l'algorisme \textit{WKNN} amb valors de K entre 6 i 8. El millors resultats dins botigues, amb \textit{MMSE} i valors alts de K.

Per altra banda, amb les dades resultants podem concloure que, sense cap dubte, la presència de clients en el centre comercial, tant en passadissos com en tendes, ha augmentat no només l'error mitjà de les estimacions de posició, sinó també la dispersió d'aquestes. Només el cas de \textit{KNN} i \textit{WKNN} per valors alts de K mostra un error mitjà major en entorns sense usuaris al voltant, i només en passadissos. En general, la major precisió en entorns amb pocs usuaris al voltant, s'aconsegueix amb l'algorisme \textit{MMSE} amb valors de K alts; en canvi, en entorns amb molts usuaris l'algorisme \textit{WKNN} amb K entre 5 i 7 proporciona el menor error mitjà.
