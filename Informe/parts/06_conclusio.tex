%Copyright (C) 2014 Sergio García Villalonga.

En l'estudi desenvolupat s'ha creat un sistema de localització en interiors aprofitant aplicacions lliures existents. A partir de la creació d'un mapa de ràdio i de dos mapes de proves s'ha estudiat la influència de la presència d'altres usuaris en la precisió del sistema.

El primer a destacar es que les xarxes WiFi permeten crear sistemes de localització en interiors d'una manera senzilla i barata. De fet, ha permès crear una en un entorn aliè a la nostra propietat en un entorn no controlat. De totes maneres, malgrat aquest avantatge, el fet de que la tecnologia WiFi no estigui dissenyada amb l'objectiu de localitzar en interiors, fa que efectes com la fluctuació del senyal, la reflexió o altres, afegeixin una gran imprecisió en les estimacions.

En el nostre cas, amb imprecisions al voltant dels 10 metres, el sistema resultant pot no ser suficient depenent de l'aplicació a la qual es vulgui orientar. Per exemple, amb aquesta imprecisió seria gairebé inútil fer-lo servir a una aplicació destinada a oferir publicitat de la tenda davant la qual es troba l'usuari. En canvi, si els comerços es trobassin agrupats per tipus d'activitat comercial (per exemple, si tots els restaurants es situassin en la mateixa zona), una aplicació podria integrar el sistema de localització aquí presentat i oferir promocions diverses.

A més, la imprecisió depèn de molts factors que, en el cas del centre comercial, poden no ser controlats pels responsables d'implementar el sistema. És el cas dels obstacles, tant estàtics com dinàmics. Entenem per obstacles estàtics la disposició de les tendes, per exemple, ja que, com s'ha demostrat, la presència de parets o estants afecta negativament, en tots els casos a l'error en la localització.

Com obstacles dinàmics, entenem l'objecte principal de l'estudi, la presència d'altres usuaris. Amb les dades resultants podem concloure que, sense cap dubte, la presència de clients en el centre comercial, tant en passadissos com en tendes, ha augmentat no només l'error mitjà de les estimacions de posició, sinó també la dispersió d'aquestes. Tot i així, en segons quins casos, l'anàlisi de les dades obtingudes mostra errors mitjans d'entre 8 i 10 metres en ambdós contexts, pel que podem afirmar que un estudi profund dels resultats depenent de l'algorisme i dels paràmetres poden permetre uns valors acceptables depenent dels requeriments d'exactitud.
