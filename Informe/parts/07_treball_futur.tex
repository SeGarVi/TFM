% Copyright (C) 2014 Sergio García Villalonga.
% This work is under the CC-BY-SA license.
% The LICENSE file contains a complete description of your rights.

Aquest estudi proposa una avaluació detallada de com afecta la presència de persones a la localització en interiors. Partint dels resultats obtinguts es proposa estendre el treball realitzat per millorar la precisió del sistema de diferents maneres:

\begin{itemize}
    \item Realitzant un estudi sobre com afecta a l'error de precisió el nombre de mostres, tant a l'hora de realitzar el mapa de ràdio, com a l'hora de realitzar el mapa de proves amb l'\textit{AirPlace Logger}. Seria útil trobar un bon ajustament de l'aplicació per millorar la precisió i comprovar que aquesta millora és aplicable a diferents contexts.
    \item Orientar un estudi similar a l'anterior però centrant-se en estudiar com hi afecta l'interval entre mostres.
    \item En el mateix sentit que la proposta anterior, també seria necessari estudiar com afecta la distància entre mostres. Una gran quantitat de mostres pot millorar la precisió del sistema final, encara que una distància massa petita entre aquestes pot obtenir valors de senyal que portin a confusió entre dos localitzacions. En aquest sentit, segurament diverses variable entren en joc, com la distància als punts d'accés, així que potser té més sentit per infraestructures i entorns controlats.
    \item Addicionalment, també seria interessant estudiar com afecta a la precisió de la localització no només la quantitat de persones al voltant de l'usuari, sinó també la quantitat de persones connectades a les xarxes \textit{WiFi} amb les que s'infereix.
    \item L'addició d'altre tipus de sistemes de mesura (no d'estimació) disponibles avui dia per millorar la precisió del sistema actual, i que no depenguessin de la presencia d'altres usuaris seria una gran millora a l'estat actual. La millora d'\textit{AirPlace} amb, per exemple, dades provinents dels sistemes inercials disponibles a la majoria de telèfons intel·ligents suposaria un aprofitament de recursos amb resultats potencialment significatius.
    \item En un altre context, l'aplicació del sistema en un context comercial també seria desitjable. Un programa per a telèfons intel·ligents que executés certes accions (com mostrar publicitat) suposaria una aproximació a una utilitat en la vida diària de molts clients. Estudiar quins requeriments seria capaç de suportar amb la precisió aquí mostrada seria una interessant proposta per un futur projecte.
\end{itemize}
