%Copyright (C) 2014 Sergio García Villalonga.

El GPS\footnote{\textit{Global Positioning System}, sistema de posicionament global} com a sistema de posicionament ha esdevingut molt precís en exteriors (entre 5 i 10 metres\cite{pogge}), però el seu principal problema és que no és efectiu en interiors. Els darrers anys s’està aprofundint en la localització d’interiors basant-se en l’ús de, per exemple, càmeres, o diverses tecnologies sense fils. Malgrat que per motius diversos la localització en interiors no ha estat tan desenvolupada al mercat com la localització en exteriors, les seves potencials aplicacions són també altament atractives. Des la ubicació en entorns més estàtics, com embalums a magatzems o maletes a aeroports, fins a situació en entorns dinàmics, com a persones dins edificis, passant per sistemes d'eficiència energètica.

Molts dels darrers estudis es centren en l’aprofitament de sistemes de senyal sense cables com puguin ser el Bluetooth\cite{kotanen} i el Wi-Fi\cite{bagosi}\cite{evennou}\cite{garcia}. De totes maneres, aquestes tecnologies no es varen dissenyar amb aquest propòsit, pel que la precisió en la localització mitjançant aquest tipus de dispositius es pot veure afectada per diversos factors. Alguns dels més comuns són les interferències de diferent procedència, la reflexió del senyals als objectes que esdevenen obstacles, o la inestabilitat de la pròpia emissió dels punts d'accés. El present estudi es centra en un dels possibles obstacles a l'hora de tenir en compte les aplicacions de localització en interiors mitjançant WiFi: la quantitat d'usuaris influeix en l’error que proporciona el sistema de localització.

Depenent de l’entorn i la seva aplicació, la gestió dels errors de localització pot ser més o menys crítica. Per exemple, podem suposar un sistema de localització en interiors per Wi-Fi de persones en un centre comercial amb la que es mostrés publicitat en tendes properes. En aquest cas, un error de posicionament pot implicar una opinió negativa del client i podria fer disminuir lleument les vendes. Un altre exemple seria l’ús del mateix sistema en una residència d’ancians, per tal de poder localitzar a cada un dels residents. En cas d’emergència, on s’hagi d’evacuar l’edifici de manera urgent, un alt marge d’error causat per la gran quantitat d'usuaris augmenta significativament la possibilitat de tragèdia, al perdre temps cercant persones en localitzacions allà on no es troben.

Per tant, el tema en el que es centra el present treball té un gran interès de cara a futurs desenvolupaments de localització en interiors en diverses situacions. A més, tota la investigació i el programari que es desenvolupi per realitzar l’estudi s’alliberaran amb llicències lliures, per tal de facilitar les investigacions futures en el mateix camp, col·laborant a un millor aprofundiment, ajudant al desenvolupament de sistemes més precisos i, a la llarga, un possible major impacte positiu en la societat.

El document s'estructura de la següent manera: la secció 2 comenta breument els objectius d'aquest estudi, en la secció 3 es presenta l'estat de l'art de les tècniques i tecnologies més utilitzades actualment en la localització en interiors, en la secció 4 s'explica com s'ha dut a terme el procés de creació del mapa de ràdio i s'exposen alguns dels problemes observats durant el desenvolupament de la tasca; la secció 5 descriu les proves realitzades per comprovar la influència de la presència d'altres usuaris a l'entorn i es comenten els resultats obtinguts; la secció 6 exposa les conclusions derivades dels resultats i, finalment, la secció 7 proposa una sèrie de possibles futurs treballs a realitzar per complementar les tesis aquí exposades.
