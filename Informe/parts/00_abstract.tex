%Copyright (C) 2014 Sergio García Villalonga.
\begin{abstract}
Els sistemes de localització en exteriors com el \textit{GPS} es troben molt desenvolupats actualment, però resulten inviables en interiors. Les diverses aplicacions comercials dels sistemes de localització en interiors els fan altament atractius. Però localitzar en interiors comporta una sèrie de problemes físics addicionals, com poden ser la presència d'obstacles. Per aquest motiu, aquest camp ha estat extensament estudiat i han sorgit nombroses aproximacions que utilitzen diferents tècniques i tecnologies per minimitzar els problemes existents.

En el cas de sistemes de localització en interiors per \textit{WiFi}, un factor que pot influir en la precisió de les estimacions és la quantitat de persones al voltant de l'usuari. Per estudiar-ne la influència, el present estudi descriu el procés de creació d'un sistema de localització en interiors que permet investigar les possibles diferències de la precisió de les localitzacions en entorns buits i amb presència d'altres persones. Posteriorment es presenten una sèrie de proves realitzades, s'exposen els seus resultats i s'evidencia la influència del nombre d'usuaris al voltant.
\end{abstract}
