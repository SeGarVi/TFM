% Copyright (C) 2014 Sergio García Villalonga.
% This work is under the CC-BY-SA license.
% The LICENSE file contains a complete description of your rights.

\thispagestyle{tfm}
\begin{abstract}
Els sistemes de localització en exteriors com el \textit{GPS}\footnote{\textit{Global Positioning System}, sistema de posicionament global} es troben molt desenvolupats actualment, però resulten inviables en interiors. Les diverses aplicacions comercials dels sistemes de localització en interiors els fan altament atractius. Però localitzar en interiors comporta una sèrie de problemes físics addicionals, com poden ser la presència d'obstacles. Per aquest motiu, aquest camp ha estat extensament estudiat i han sorgit nombroses aproximacions que utilitzen diferents tècniques i tecnologies per minimitzar els problemes existents.

En el cas de sistemes de localització en interiors per \textit{WiFi}, un factor que pot influir en la precisió de les estimacions és la quantitat de persones al voltant de l'usuari. Per estudiar-ne la influència, el present estudi descriu el procés de creació d'un sistema de localització en interiors que permet investigar les possibles diferències de la precisió de les localitzacions en entorns buits i amb presència d'altres persones. Posteriorment es presenten una sèrie de proves realitzades, s'exposen els seus resultats, així com les conclusions sobre com influencia del nombre d'usuaris a la precisió del sistema.
\newline

\footnotesize

Outdoor location systems like GPS are notably developed, but they become inviable within buildings. The potential applications of indoor location are very appealing, but it also involves additional physical problems, as the presence of obstacles. Due to these difficulties, this field has been extensively studied and several solutions have been published. They use different methodologies and technologies to try to minimize the existing physical problems.

In the concrete case of indoor localization by WiFi, one of the factors that can influence on the precision of estimations is the amount of people in the environment. The present study describes the creation process of an indoor location system that allows the investigation of possible differences in accuracy between empty and crowded environments. It also presents the completed tests and obtained results, as well as the derived conclusions on how the amount of users affects the precision of the system.
\normalsize
\newline

\end{abstract}
